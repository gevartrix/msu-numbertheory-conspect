%%% PREAMBLE :: Layout
%
% Sets the general layout of the document (margins, spacing, title format, section format, headers, footers, referencing style, etc.)


%% Imports required packages
% Provides more granular control over the layout of `enumerate', `itemize' and `description'
\usepackage[shortlabels]{enumitem}

% Sets the pages' headers and footers for the main part of the document
\usepackage{fancyhdr}

% Sets the pages' layout
\usepackage[%
    a4paper,%         Same as the size of the article
    hmargin=3.175cm,% Equivalent to `left=3.175cm, right=3.175cm`
    vmargin=2.54cm%   Equivalent to `top=2.54cm, bottom=2.54cm`
]{geometry}

% Provides extensive handling and customization of cross-referencing and creating hypertext links
\usepackage[%
    pdfusetitle%  Catches the document's @title and @author arguments for \hypersetup 
]{hyperref}

% Enables conditional commands like `\ifthenelse` and `\whiledo` 
\usepackage{ifthen}

% Indents first paragraph after (sub)section headers
\usepackage{indentfirst}

% Allows reference to the document's last page's number (required for the fancyhdr footer) 
\usepackage{lastpage}

% Sets up easily configurable ragged text environment (required for the `lecture` environment to display the headers and footers properly)
\usepackage{ragged2e}

% Edits the layout of sections, subsections, etc.
\usepackage[%
    compact,% Reduces the spacing above and below the titles
    sc,%      Sets the default font shape to `scientific`
    center%   Aligns to the center of a page
]{titlesec}

% Provides an underline command which breaks over line ends. Imported for underling URLs (see the `\myurl` command)
\usepackage[%
    normalem% Disables redefining of `\em` and `\emph`
]{ulem}


%% Sets up general indentation
\setlength{\parindent}{2.3em}%         Paragraph indentation
%\setlength{\parskip}{0.2cm}%           Spacing between paragraphs
%\renewcommand{\baselinestretch}{1.4}%  Line spacing


%% Handles with overfulls and page breaks
% Changes the setting of paragraphs that would have produced over-full boxes, tries an extra pass with pretending that every line has an additional bit of space. Prevents the overall badness exceeding 1000
\setlength{\emergencystretch}{1.5em}
\clubpenalty=10000   % Prohibits page-breakings after just one paragraph line
\widowpenalty=10000  % Prohibits page-breakings after last paragraph lines


%% Sets up the lists
% [enumitem] Defines the `statesp` and `casesp` lists 
\newlist{statesp}{enumerate}{1}
\newlist{casesp}{enumerate}{2}

% [enumitem] Makes the `enumerate` list more compact
\setenumerate{%
    nosep,%            Removes vertical spacing
    nolistsep,%        Removes spacing between the list itself and the neighbouring paragraphs
    noitemsep,%        Removes spacing between items and paragraphs
    topsep=0pt,%       Removes spacing between first item and preceding paragraph
    parsep=0pt,%       Removes spacing between paragraphs within an item
    partopsep=0pt,%    Removes extra spacing added to `\topsep`
    label=(\arabic*)%  Sets the default label
}

% [enumitem] Makes the `itemize` list more compact
% Sets the default label for the first level
\renewcommand{\labelitemi}{$\Diamond$}
\setlist[itemize]{%
    nosep,%                 Removes vertical spacing
    nolistsep,%             Removes spacing between the list itself and the neighbouring paragraphs
    noitemsep,%             Removes spacing between items and paragraphs
    topsep=0pt,%            Removes spacing between first item and preceding paragraph
    parsep=0pt,%            Removes spacing between paragraphs within an item
    partopsep=0pt,%         Removes extra spacing added to `\topsep`
}

% [enumitem] Sets the `statesp` list
\setlist[statesp]{%
    align=left,%                Aligns the labels properly
    nosep,%                     Removes vertical spacing
    nolistsep,%                 Removes spacing between the list itself and the neighbouring paragraphs
    noitemsep,%                 Removes spacing between items and paragraphs
    topsep=0pt,%                Removes spacing between first item and preceding paragraph
    parsep=0pt,%                Removes spacing between paragraphs within an item
    partopsep=0pt,%             Removes extra spacing added to `\topsep`
    font=\normalfont\bfseries%  Boldens the labels
}
\setlist[statesp,1]{%
  label=Пункт~(\arabic*):,%     Sets label format for the top level
  ref=\arabic*%                 Sets cross-ereferencing format for the top level
}

% [enumitem] Sets the `casesp` list
\setlist[casesp]{%
    align=left,%
    nosep,%
    noitemsep,%
    parsep=0pt,%
    partopsep=0pt,%
    font=\normalfont\bfseries%
}
\setlist[casesp,1]{%
    label=Случай~\arabic*:,%
    ref=\arabic*%
}
\setlist[casesp,2]{%
    label=Случай~\thecasespi.\roman*:,%
    ref=\thecasespi.\roman*%
}


%% [fancyhdr, ifthen, lastpage] Sets up the header and footer parts of pages' layout
% Clears header and footer completely
\fancyhf{}
% Puts the current subsection number and name on the left side of the header
% If subsection has been reset by a new section it puts the current section number and name instead
% The `\leftmark` command outputs the number and name of the current section in the `article` document class, while the `\rightmark` does it for current subsection
\lhead{
    \scshape\small\ifthenelse{%
        \equal{\rightmark}{}}%
        {\bfseries\nouppercase{\leftmark}}%
        {\S\rightmark}
}
% Puts the current lecture number on the right side of the header
\rhead{
    \it{Лекция~\textnumero\themylecture}
}
% Puts the pae counter on the right side of the footer
\rfoot{
    Стр. \thepage~/ \pageref*{LastPage}
}
% Tweaks the width of the top rule
\renewcommand{\headrulewidth}{0.1pt}
% Adjusts the topmargin for fancyhdr
\setlength{\headheight}{14pt}
% Defines an alternative page style for the sections' supplementary material
\fancypagestyle{supplementary}{%
    \lhead{
        \scshape\small\bfseries\nouppercase{\leftmark}
    }
}


%% Sets up the linking style and the PDF-specific options
% [hyperref] Overrides the `\and` command while creating the document's metadata (in case of several authors)
\pdfstringdefDisableCommands{\def\and{и}}
% [hyperref] Sets up links' style and metadata information
\makeatletter%  Changes the `@` symbol's category from `other` (catcode: 12) to `letter` (catcode: 11), so it's treated as a normal letter
\DeclareUrlCommand\myurl@@{%
    \def\UrlFont{\color{blue}}%
    \def\UrlLeft{\uline\bgroup}%
    \def\UrlRight{\egroup}
}
\def\myurl@#1#2{\hyper@linkurl{\myurl@@{#2}}{#1}}
\DeclareRobustCommand*\myurl{\myurl@}
\hypersetup{%
    linktocpage=true,%                   Makes the page numbers from the Table Of Contents interactive
    colorlinks=true,%                    Colours the (cross-referencing) links
    linkcolor=[rgb]{0.84, 0.18, 0.14},%  Sets the colour of the cross-referencing links (sections, Theorems, Lemmas, etc.)
    citecolor=[rgb]{0.1, 0.54, 0.24},%   Sets the colour of citations to the supplemantary mmaterial (and bibliography)
    urlcolor=[rgb]{0.1, 0.34, 0.68},%    Sets the colour of the web addresses
    bookmarksopen=true,%                 Makes the bookmark bar open by default in a PDF reader
    pdfpagemode=FullScreen,%             Makes a PDF reader open the document in full screen
    pdfauthor={\@author},%               Sets the document's author(s) based upon the `\author`
    pdftitle={\@title}%                  Sets the document's title based upon the `\title`
}
\makeatother%  Changes the `@` symbol's category from `letter` (catcode: 11) to `other` (catcode: 12), so it could be safely used in macros


%% [titlesec] Sets up the format of sections and subsections
% Converts the sections' labels to the Roman numerals
\renewcommand{\thesection}{\Roman{section}}
\titleformat%
    {\section}%                           Redefines the sectioning command (`\section`)
    [block]%                              Sets paragraph's shape so a section's label and title are on the same line
    {\centering\Large\scshape\bfseries}%  Sets label's and title's format
    {\thesection.}%                       Sets label
    {0.3cm}%                              Sets the spacing between label and title body
    {}%                                   Sets the after-code
\titleformat%
    {\subsection}%
    [block]%
    {\centering\large\scshape}%
    {\S\thesubsection.}%
    {0.3cm}%
    {}
% Adjusts the vertical spacing for the sections' titles
\titlespacing*%
    {\section}%   Sets spacing around the sections' titles
    {0cm}%        Sets the left margin
    {1cm}%        Sets the vertical space before the title
    {0.5cm}%      Sets the vertical space after the title
\titlespacing*%
    {\subsection}%
    {0cm}%
    {0.4cm}%
    {0.2cm}
% Starts each section (except the first one) on a new page
% N.B.: Depends on the definition of `\thesection` above
\newcommand{\sectionbreak}{
    \ifthenelse{%
        \equal{\thesection}{I}}%  If section's number is "I"
        {}%                       Skip
        {\clearpage}%             Else start the section on a new page
}


%% [hyperref, ragged2e] Defines an environment that sets up the lectures' header and footer. Takes one mandatory argument -- the lecture's date -- and one optional argument -- the lecture's number. If the optional argument is passed, it updates a counter
% Initializes a counter for lecture numbers
\newcounter{mylecture}
\newenvironment{lecture}[2][\themylecture]%
    {%
        % If the optional argument isn't passed, increment the default counter
        \refstepcounter{mylecture}
        \vspace{0.15cm}
        % Since arguments passed to an environment are only available for the start-code, a hack with storing these arguments is used
        % Defines the lecture's number as the optional argument if it's been passed, or as the default counter's value otherwise. `\def` is used to overwrite the commands each time
        \def\lecnum{#1}
        \def\lecdate{#2}

        % Sets the environment's header
        \begin{FlushRight}
            \scriptsize
            \textit{Лекция \textnumero\lecnum~от \lecdate}
            \label{lec:no#1}
            \hrule
        \end{FlushRight}
    }%
    {%
        % Updates the counter, so the next lecture's number agrees.
        \setcounter{mylecture}{\lecnum}

        % Sets the environment's footer
        \begin{FlushLeft}
            \hrule
            \scriptsize
            \textit{\hyperref[lec:no\lecnum]{Вернуться} к началу Лекции \textnumero\lecnum.}
        \end{FlushLeft}
    }
