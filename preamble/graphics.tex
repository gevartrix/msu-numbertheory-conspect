%%% PREAMBLE :: Graphics
%
% Sets up everything for creating, importing and working with various graphics in the document


%% Imports required packages
% Allows to import, create and manipulate graphics
% \usepackage{graphics}

% Extends the `graphics` package. Allows optional arguments according to the "key=value" scheme. Highly recommended over the `graphics` package
\usepackage{graphicx}

% Provides the tools necessary to create graphic elements (plots, graphs, diagrams) in LaTeX itself
\usepackage{tikz}


%% Sets up paths to the assets
% [graphicx] Points to the folders to search images in.
\graphicspath{%
    {././img/}%  Assuming this file is in the root's "preamble/" folder, it points to the root's "img/" folder
}

%% Sets up required TikZ libraries
% [tikz] Allows for more precise coordinate calculations
\usetikzlibrary{calc}
% [tikz] Enables varios markings at any place of a line
\usetikzlibrary{decorations.markings}
% [tikz] Defines various geometric shapes other than rectangle, circle and co-ordinate.
% Allows to draw fancy hyperplanes (parallelograms) in Lecture 10 (see img/figures/10-2.tikz)
\usetikzlibrary{shapes.geometric}
