%%% PREAMBLE :: Mathematical packages
%
% Loads a set of packages developed for the American Mathematical Society
% Sets up theorem environments and additional commands/operators
% Essential for compiling any mathematical research paper


%% Imports required packages
% Provides enhancements for properly outputing mathematical formulas
\usepackage{amsmath}

% Internally loads the `amsfonts` package, provides an extended symbol collection
\usepackage{amssymb}

% Provides support of various environments (Theorem, Lemma, proof, etc.)
\usepackage{amsthm}

% Enables support of commutative diagrams
% \usepackage{amscd}

% Enables support of blackboard bold letters, Fraktur letters and miscellaneous symbols
% \usepackage{amsfonts}


%% Defines various maths environments (Theorem, Lemma, proof, etc.). Both numbered and unnumbered
% [amsthm] Defines the Theorem-style environments
\theoremstyle{plain}
\newtheorem{ntheorem}{Теорема}[section]
\newtheorem{nlemma}{Лемма}[section]
\newtheorem{nproposition}{Предложение}[section]
\newtheorem{ncorollary}{Следствие}[section]
\newtheorem*{theorem}{Теорема}
\newtheorem*{lemma}{Лемма}
\newtheorem*{proposition}{Предложение}
\newtheorem*{corollary}{Следствие}
\newtheorem*{hypothesis}{Гипотеза}
\newtheorem*{test}{Признак}

% [amsthm] Defines the Definition-style environments
\theoremstyle{definition}
\newtheorem{ndefinition}{Определение}[section]
\newtheorem{nstatement}{Утверждение}[section]
\newtheorem{nexample}{Пример}[section]
\newtheorem*{definition}{Определение}
\newtheorem*{statement}{Утверждение}
\newtheorem*{example}{Пример}

% [amsthm] Defines the Remark-style environments
\theoremstyle{remark}
\newtheorem{nproblem}{Задача}[section]
\newtheorem{nremark}{Замечание}[section]
\newtheorem*{problem}{Задача}
\newtheorem*{remark}{Замечание}

% [amsthm, amssymb] Redefines the proof's QED symbol
\renewcommand{\qedsymbol}{$\blacksquare$}

% [amsthm, amssymb] Defines the `Solution` environment
\newenvironment{solution}
    {%
        \renewcommand{\qedsymbol}{$\square$}%
        \begin{proof}[Решение]%
    }
    {\end{proof}}


%% Defines and redefines common mathematical operators
% [amsfonts] Defines commonly used sets (fields)
\newcommand{\field}[1]{\ensuremath{\mathbb{#1}}}
\newcommand{\CC}{\field{C}}    % Complex numbers
\renewcommand{\AA}{\field{A}}  % Algebraic numbers
\newcommand{\RR}{\field{R}}    % Real numbers
\newcommand{\QQ}{\field{Q}}    % Rational numbers
\newcommand{\ZZ}{\field{Z}}    % Integer numbers
\newcommand{\NN}{\field{N}}    % Natural numbers

% [amsmath] Changes `\Re` and `\Im` from Fraktur to Roman and puts its argument in parentheses
\renewcommand{\Re}[1]{\operatorname{Re}\left( #1 \right)}
\renewcommand{\Im}[1]{\operatorname{Im}\left( #1 \right)}

% [amsmath] Redefines the predefined operators of AMSMATH to Russian notation and puts its argument in parentheses (e.g. sin x => sin(x), tan x => tg(x), etc.)
% If any default operators are needed, the corresponding lines can be commented out
% Trigonometric functions
\renewcommand{\sin}[1]{\operatorname{sin}\left( #1 \right)}
\renewcommand{\cos}[1]{\operatorname{cos}\left( #1 \right)}
\renewcommand{\tan}[1]{\operatorname{tg}\left( #1 \right)}
\renewcommand{\cot}[1]{\operatorname{ctg}\left( #1 \right)}
\renewcommand{\arcsin}[1]{\operatorname{arcsin}\left( #1 \right)}
\renewcommand{\arccos}[1]{\operatorname{arccos}\left( #1 \right)}
\renewcommand{\arctan}[1]{\operatorname{arctg}\left( #1 \right)}
\renewcommand{\sec}[1]{\operatorname{sec}\left( #1 \right)}
\renewcommand{\csc}[1]{\operatorname{cosec}\left( #1 \right)}
\renewcommand{\sinh}[1]{\operatorname{sh}\left( #1 \right)}
\renewcommand{\cosh}[1]{\operatorname{ch}\left( #1 \right)}
\renewcommand{\tanh}[1]{\operatorname{th}\left( #1 \right)}
\renewcommand{\coth}[1]{\operatorname{cth}\left( #1 \right)}
% Logarithms
\renewcommand{\log}[1]{\operatorname{log}\left( #1 \right)}
\renewcommand{\ln}[1]{\operatorname{ln}\left( #1 \right)}
\renewcommand{\lg}[1]{\operatorname{lg}\left( #1 \right)}
% Limits
\renewcommand*{\limsup}{\varlimsup}
\renewcommand*{\liminf}{\varliminf}
% Defines a limit operator with both upper and lower bars
\DeclareMathOperator*{\limsi}{\overline{\underline{lim}}}
\newcommand*{\limsupinf}[1]{\limsi\left( #1 \right)}
% Miscellaneous
\renewcommand{\deg}[1]{\operatorname{deg}\left( #1 \right)}
\renewcommand{\det}[1]{\operatorname{det}\left( #1 \right)}
\renewcommand{\dim}[1]{\operatorname{dim}\left( #1 \right)}
\renewcommand{\exp}[1]{\operatorname{exp}\left( #1 \right)}
\renewcommand{\gcd}[1]{\textrm{НОД}\left( #1 \right)}
\renewcommand{\hom}[1]{\operatorname{Hom}\left( #1 \right)}
\renewcommand{\ker}[1]{\operatorname{Ker}\left( #1 \right)}

% [amssymb] Redefines the `greater than` and `less than` signs
\renewcommand{\ge}{\geqslant}
\renewcommand{\le}{\leqslant}

% Redefines the look of some Greek letters commonly used as variables or functions
\renewcommand{\epsilon}{\varepsilon}
\renewcommand{\phi}{\varphi}

% Redefines the universal and existential quantifiers to adjust spacing
\let\oldforall\forall
\let\oldexist\exists
\renewcommand{\forall}{\oldforall \, }
\renewcommand{\exists}{\oldexist \: }
\newcommand\existu{\oldexist! \: }


%% Defines additional commands for document-specific notations
% Defines the `divisible by` symbol (three vertical dots with the right spacing)
\DeclareRobustCommand{\divisibleby}{%
    \mathrel{%
        \text{%
            \vbox{%
                \baselineskip.65ex%
                \lineskiplimit0pt%
                \hbox{.}\hbox{.}\hbox{.}%
            }%
        }%
    }%
}
% Provides flexible vertical bars for `\abs`
% It measures how much larger the content is allowed to get before the delimiters start growing, but by setting it to a negative value delimiters always grow
\delimitershortfall-1sp
\newcommand{\abs}[1]{\ensuremath{\left| #1 \right|}}
\newcommand{\lcd}[1]{\textrm{НОК}\left( #1 \right)}
\newcommand{\congr}[3]{\ensuremath{#1 \equiv #2 \pmod #3}}
\newcommand{\notcongr}[3]{\ensuremath{#1 \not\equiv #2 \pmod{#3}}}
\newcommand{\sfrac}[2]{\ensuremath{#1 / #2}}
\newcommand{\divides}{\mid}
\newcommand{\notdivides}{\nmid}
\newcommand{\divby}{\divisibleby}
\newcommand{\notdivby}{\not\divby}
\newcommand{\1}{\textbf{1}}
\DeclareMathOperator*{\Res}{Res}
\newcommand{\bigO}[1]{\ensuremath{\mathcal{O}\left( #1 \right)}}


%% Toggles the display style to be equation-like everywhere
% It's generally not recommended, as it increases the spacing between the lines and tears the standard baseline. But in our case it improves the document's readability
\everymath{\displaystyle}
