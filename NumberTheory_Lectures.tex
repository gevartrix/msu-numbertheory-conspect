\documentclass[11pt]{article}
\usepackage[utf8]{inputenc}	
\usepackage[T2A]{fontenc}
\usepackage[english,russian]{babel}
\usepackage{amsmath,amsthm,amsfonts,amssymb,amscd}
\usepackage{multirow,booktabs}
\usepackage[table]{xcolor}
\usepackage{fullpage}
\usepackage{lastpage}
\usepackage{enumitem}
\usepackage{fancyhdr}
\usepackage{mathrsfs}
\usepackage{wrapfig}
\usepackage{setspace}
\usepackage{calc}
\usepackage{multicol}
\usepackage{cancel}
\usepackage[margin=3cm]{geometry}
\usepackage{amsmath}
\newlength{\tabcont}
\setlength{\parindent}{0.0in}
\setlength{\parskip}{0.05in}
\usepackage{empheq}
\usepackage{import}
\usepackage{framed}
\usepackage[most]{tcolorbox}
\usepackage{xcolor}
\usepackage[space]{grffile}
\usepackage{graphicx}
\usepackage{hyperref}
\graphicspath{{./img/}}
\colorlet{shadecolor}{orange!15}
\parindent 0in
\parskip 12pt
\geometry{margin=1in, headsep=0.25in}
\newcommand\tab[1][1cm]{\hspace*{#1}}

\DeclareMathOperator{\Ree}{Re}
\DeclareMathOperator{\Imm}{Im}

\textwidth=500pt
\textheight=700pt
\oddsidemargin=20pt
\hoffset=-1.5cm
\topmargin=-25mm

\theoremstyle{plain}
\newtheorem{theorem}{Теорема}[section]
\newtheorem{hypothesis}{Гипотеза}[section]
\newtheorem{lemma}{Лемма}[section]
\newtheorem{proposition}{Предложение}[section]
\newtheorem{corollary}{Следствие}[section]
\newtheorem*{note}{Замечание}
\newtheorem{problem}{Задача}[section]
\newtheorem*{test}{Признак}

\theoremstyle{definition}
\newtheorem{definition}{Определение}[section]
\newtheorem{example}{Пример}[section]
\newtheorem{statement}{Утверждение}[section]

\theoremstyle{remark}
\newtheorem*{rem}{Замечание}

\newenvironment{pf}
  {\par\noindent{\it Доказательство.}}
  {\hfill$\scriptstyle\blacksquare$~\\}
  
\newenvironment{solution}
  {\renewcommand\qedsymbol{$\blacksquare$}\begin{proof}[Решение]}
  {\end{proof}}

\renewcommand{\leq}{\leqslant}
\renewcommand{\geq}{\geqslant}

\renewcommand{\phi}{\varphi}
\renewcommand{\epsilon}{\varepsilon}

\renewcommand{\Re}{\Ree}
\renewcommand{\Im}{\Imm}

\hypersetup{
    colorlinks,
    citecolor=green,
    filecolor=red,
    linkcolor=red,
    urlcolor=blue
}


\begin{document}
\title{Лекции по теории чисел}


\thispagestyle{empty}

\import{./src/}{Title}

\newpage
~\\~\\~\\
\import{./src/}{Annotation}

%\begin{center}
	% \Large За\TeX ано Артемиями, Геворковым и Соколовым.\\~\\
%\end{center}

\newpage

\tableofcontents

\newpage

%\import{./src/}{Summary}
%\newpage

%Лекция 01 (05.9.2017)
\import{./src/}{Lecture01}

%Лекция 02 (12.9.2017)
\import{./src/}{Lecture02}

%Лекция 03 (19.9.2017)
\import{./src/}{Lecture03}

%Лекция 04 (26.9.2017)
\import{./src/}{Lecture04}

%Лекция 05 (03.10.2017)
\import{./src/}{Lecture05}

%Лекция 06 (10.10.2017)
\import{./src/}{Lecture06}

%Лекция 07 (17.10.2017)
\import{./src/}{Lecture07}

%Лекция 08 (24.10.2017)
\import{./src/}{Lecture08}

%Лекция 09 (31.10.2017)
\import{./src/}{Lecture09}

%Лекция 10 (07.11.2017)
\import{./src/}{Lecture10}

%Лекция 11 (14.11.2017)
\import{./src/}{Lecture11}

%Лекция 12 (21.11.2017)
\import{./src/}{Lecture12}

%Лекция 13 (28.11.2017)
\import{./src/}{Lecture13}

%Лекция 14 (05.12.2017)
\import{./src/}{Lecture14}

\end{document}