% Sets the page numbering style to Roman lowercase, removing the ambiguity for hyperref
\pagenumbering{roman}


\section*{Предисловие}


\itshape

\textbf{Disclaimer!} Это не курс лекций и не методичка, а всего лишь конспект курса, набранный в системе компьютерной вёрстки \LaTeXe~и не претендующий на истину в последней инстанции. Не исключено также, что в данной версии присутствуют неточности и опечатки, хотя мы --- авторы --- и старались свести их количество к минимуму. В любом случае этот документ предлагается использовать на свой страх и риск, и мы не несём ответственности за успешность подготовки по данному материалу, а также за его использование в качестве ``шпоры'' на коллоквиумах или (боже упаси!) экзаменах.

Конспект состоит из 14-ти лекций, прочитанных Олегом Николаевичем Германом --- доцентом кафедры теории чисел мехмата МГУ. Курс был прочитан в 7-ом семестре мехмата осенью 2017 года и состоит из трёх основных частей:
\begin{enumerate}
    \item Асимптотический закон распределения простых чисел:
        % !IMPORTANT!
        % The `\[...\]` syntax should always be used over `$$...$$` when working in LaTeX, as it doesn't screw up the vertical spacing
        \[
            \pi(x) = \sum_{p \le x} 1 \sim \frac{x}{\ln{x}}.
        \]
    \item Теорема Дирихле о простых числах в арифметических прогрессиях: если ${\gcd{l,\, m} = 1}$, то существует бесконечное количество таких простых $p$, что ${\congr{p}{l}{m}}$.
    \item Теоремы о том, что $e$ и $\pi$ — иррациональные и трансцендентные числа.
\end{enumerate}

Этот конспект был подготовлен и за\TeXан студентами Артемием Соколовым, группа 405 (нечётные лекции) и Артемием Геворковым, группа 402 (чётные лекции) ещё в декабре 2017 года. За основу нами был взят конспект Юлии Зайцевой.

В декабре 2020 года я --- Артемий Геворков --- решил вернуться к этому документу, чтобы улучшить качество вёрстки, подправить некоторые неточности и привести сам код в порядок (с учётом всего моего накопленного опыта по вёрстке в системе \LaTeXe~--- оформления конспектов, домашек, курсовых и т.д. --- за все годы обучения на мехмате). Кроме того, я снабдил некоторые доказательства пояснительными рисунками, которые были нарисованы с помощью пакета~\texttt{tikz}. Также я переписал б\'{о}льшую часть кода преамбулы документа, снабдив сам код подробными комментариями. Если кому-то из читателей понравится данная вёрстка, и он или она захочет оформить свой конспект (или любой другой документ) в подобной вёрстке, то, надеюсь, этих комментариев будет достаточно, чтобы разобраться в коде и настроить его под свои нужды.

В перспективе я ещё планирую добавить решения всех упражнений из данного курса лекций, а также решения ряда заданий, разобранных на семинарских занятиях Олега Николаевича.

Данная версия документа была скомпилирована \today~Последняя версия конспекта вместе со всеми его исходными файлами всегда будет доступна в этом~\href{https://github.com/gevartrix/msu-numbertheory-conspect}{Github--репозитории} (ссылка кликабельна). Если вы нашли в самой свежей версии документа какую-нибудь неточность, или если у вас есть какие-либо вопросы/замечания --- пожалуйста, откройте в репозитории issue или присылайте pull request.

Спасибо Юлии Зайцевой, Виталию Лобачевскому, Всеволоду Гусеву, Кириллу Сосову, Сергею Джунусову, Айку Эминяну, Александру Думаревскову и команде Алгебрача за поиск ошибок и помощь в оформлении данного материала.

\normalfont
\newpage
