% Makes the titlepage's layout wider
\newgeometry{
    hmargin=2cm,
    vmargin=1.8cm
}

% Adds an anchor for 'hyperref' in order to get to the correct page from the ToC
\clearpage
\phantomsection

% Sets the fancyhdr pagestyle for the questions section
\fancyhfoffset[O]{0pt}
\pagestyle{questions}


\section*{Вопросы к экзамену}
% Adds this unnumbered section to the document's Table of Contents
\label{sec:exam-questions}
\addcontentsline{toc}{section}{Вопросы к экзамену}

{\centering\scshape
    \huge
    ТЕОРИЯ ЧИСЕЛ

    \vspace{0.5cm}

    \large
    4 курс, 1 поток, 2017--2018 г.

    \vspace{0.15cm}

    \normalsize
    лектор: д.ф.--м.н. Герман О.Н.

    \vspace{1cm}
\par}

\begin{enumerate}[%
  label={\arabic*.},%  Sets the labels ('1.', '2.', etc.)
  itemsep=4pt,%        Sets spacing between items
  parsep=2pt%          Sets spacing between paragraphs within items
]
    \item  % 01
        Оценки Чебышёва для функции $\pi(x)$. 
        Доказательство равносильности асимптотического закона распределения простых чисел утверждению $\lim_{x \to +\infty} \left( \frac{\psi(x)}{x} \right) = 1$.
    \item  % 02
        Определение функции $\zeta(s)$ и её простейшие свойства в области $\Re{s} > 1$: аналитичность, представление $\zeta(s)$ и $\frac{\zeta'(s)}{\zeta(s)}$ в виде рядов Дирихле, отсутствие нулей.
    \item  % 03
        Тождество Эйлера для $\zeta(s)$.
    \item  % 04
        Аналитическое продолжение $\zeta(s)$ в область $\Re{s} > 0$.
    \item  % 05
        Отсутствие нулей у функции $\zeta(s)$ на прямой $\Re{s} = 1$.
    \item  % 06
        Вывод утверждения $\lim_{x \to +\infty} \left( \frac{\psi(x)}{x} \right) = 1$ из сходимости интеграла $\int_{1}^{+\infty} \frac{\psi(x) - x}{x^2}dx$.
    \item  % 07
        Доказательство сходимости интеграла $\int_{1}^{+\infty} \frac{\psi(x) - x}{x^2}dx$.
    \item  % 08
        Характеры Дирихле. Вычисление сумм $\sum_a \chi(a)$ и $\sum_\chi \chi(a)$. 
        Доказательство неравенства $\abs{\sum_{n=1}^{m} \chi(n)} \le \phi(m)$ для неглавного характера.
    \item  % 09
        $L$--функции Дирихле и их простейшие свойства в области $\Re{s} > 1$: аналитичность, представление $L'(s, \chi)$ и $\frac{L'(s, \chi)}{L(s, \chi)}$ в виде рядов Дирихле, отсутствие нулей.
    \item  % 10
        Тождество Эйлера для $L$--функций, аналитическое продолжение $L(s, \chi)$ в область ${\Re{s} > 0}$.
    \item  % 11
        Доказательство утверждения $L(1, \chi) \ne 0$ для неглавных вещественных характеров $\chi$.
    \item  % 12
        Доказательство утверждения $L(1, \chi) \ne 0$ для невещественных характеров $\chi$.
    \item  % 13
        Доказательство Теоремы Дирихле о простых числах в арифметической прогрессии.
    \item  % 14
        Теорема Дирихле о приближении действительных чисел рациональными. 
        Следствие для иррациональных чисел.
    \item  % 15
        Теорема Лиувилля о приближении рациональными числами алгебраических чисел. 
        Иррациональность и трансцендентность числа $\sum_{n=0}^{\infty} 2^{-n!}$.
    \item  % 16
        Иррациональность чисел $e$ и $\pi$.
    \item  % 17
        Трансцендентность числа $e$.
    \item  % 18
        Алгебраические числа. 
        Замкнутость множества алгебраических чисел относительно арифметических операций.
    \item  % 19
        Целые алгебраические числа. 
        Замкнутость множества целых алгебраических чисел относительно сложения, вычитания и умножения.
    \item  % 20
        Теорема о примитивном элементе. 
        Степень конечного расширения поля \QQ.
    \item  % 21
        Алгебраическая замкнутость поля алгебраических чисел.
    \item  % 22
        Вложения конечного расширения поля \QQ~в поле \CC. 
        Нормальные расширения. 
        Определение группы Галуа.
    \item  % 23
        Множество образов элемента при различных вложениях конечного расширения. 
        Норма в алгебраическом расширении.
    \item  % 24
        Теорема Линдемана--Вейерштрасса. 
        Следствия из неё. 
        Сведение доказательства к Теореме об экспоненциальной линейной форме, коэффициенты ряда Тейлора которой рациональны.
    \item  % 25
        Доказательство Теоремы об экспоненциальной линейной форме, коэффициенты ряда Тейлора которой рациональны.
\end{enumerate}

\vspace{1cm}

\begin{FlushRight}
    \small\itshape
    Желаем продуктивного бота и лайтовых вопросов на экзамене! :D
\end{FlushRight}


\restoregeometry
