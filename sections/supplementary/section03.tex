\thispagestyle{supplementary}

% Encapsulates in a group, so it becomes possible to temporary redefine `thebibliography`'s header macros
% (\section -> \subsection)
\begingroup
    \renewcommand{\section}[2]{\subsection#1{#2}}%

    \begin{thebibliography}{5}
        \bibitem[ГНШ1984]{GNSh1984}
            Галочкин А.И., Нестеренко Ю.В., Шидловский А.Б. 
            \textit{Введение в теорию чисел, 4.\S4}. 
            Издательство МГУ, 1984.

        \bibitem[Бугаенко2001]{Bugaenko2001}
            Бугаенко В.О. 
            \href{https://www.mccme.ru/mmmf-lectures/books/books/book.13.pdf}{\textit{Уравнения Пелля}}. 
            Издательство МЦНМО, 2001.~\newline
            ISBN: 5-900916-96-0.

        \bibitem[Moshchevitin2012]{Moshchevitin2012}
            Nikolay G. Moshchevitin. 
            \href{https://arxiv.org/pdf/1202.4539.pdf}{\textit{On some open problems in Diophantine approximation}}. 
            2012.~\newline
            ArXiv: \href{https://arxiv.org/abs/1202.4539}{[arXiv:1202.4539]}.

        \bibitem[Мощевитин2020]{Moshchevitin2020}
            Мощевитин Н.Г. 
            \href{http://www.mathnet.ru/present26192}{\textit{Диофантовы приближения: осцилляция и изоляция}}. 
            Видеозапись доклада на конференции ``Классическая механика, динамические системы и математическая физика''. 
            2020.

        \bibitem[Ishak2008]{Ishak2008}
            Daniel Ishak. 
            \href{https://uu.diva-portal.org/smash/get/diva2:302893/FULLTEXT01.pdf}{\textit{The Thue--Siegel--Roth Theorem}}. 
            2008.
    \end{thebibliography}
\endgroup
