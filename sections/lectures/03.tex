\subsection{Результаты из Преобразования Абеля}
\label{subsec:I-4}

\begin{nlemma}[Преобразование Абеля]
\label{lm:I-6}
    Пусть $\{ a_n \}_{n \in \NN}$ --- последовательность комплексных чисел. 
    Пусть $g(x)$ --- $\CC$--значная функция и $g(x) \in C^1\left( \left[ 1, +\infty \right), \mathbb{C} \right)$. 
    Пусть $A(x) = \sum_{n \le x} a_n$.
    Тогда для любого $N \in \RR$ справедливо
    \[
        \sum_{n \le N} a_n g(n) = A(N)g(N) - \int_{1}^{N} A(x)g'(x)dx.
    \]
\end{nlemma}
\begin{proof}
    \begin{align*}
        A(N)g(N) - \sum_{n \le N} a_n g(n) &= \sum_{n \le N} a_n(g(N) - g(n)) \\
        &= \sum_{n \le N} a_n\int_{n}^{N} g'(x)dx.
    \end{align*}
    Введём теперь функцию $\phi_n(x) := 
    \begin{cases}
        a_n, & x \ge n, \\ 
        0, & x < n 
    \end{cases}$. Тогда
    \begin{align*}
        \sum_{n \le N} a_n\int_{n}^{N} g'(x)dx &= \sum_{n \le N} \int_{1}^{N}\phi_n(x)g'(x)dx \\
        &= \int_{1}^{N} \left( \sum_{n \le N} \phi_n(x) \right) g'(x)dx \\
        &= \int_{1}^{N} A(x)g'(x)dx.
    \end{align*}
    Последнее равенство верно, так как 
    \[
        \sum_{n \le N} \phi_n(x) =
        \sum_{n \le x} \phi_n(x) =
        \sum_{n \le x} a_n = A(x).
    \]
\end{proof}

\begin{ntheorem}
\label{thm:I-5}
    \[
        \zeta(s) = 1 + \frac{1}{s - 1} - s \int_{1}^{+\infty} \frac{\{x\}}{x^{1+s}}dx,
    \]
    причём интеграл в правой части сходится в полуплоскости $\Re{s} > 0$ и задаёт аналитическую функцию.
\end{ntheorem}
\begin{proof}
    При $\Re{s} > 1$ выполнено $\zeta(s) = \sum_{n = 1}^{\infty} \frac{1}{n^s}$. Используя Лемму~\ref{lm:I-6} с параметрами $a_n = 1$ и $g(x) = \frac{1}{x^s}$, получаем
    \begin{align*}
        \sum_{n = 1}^{N} \frac{1}{n^s} 
        &= N \frac{1}{N^s} + s \int_{1}^{N}\frac{\left[ x \right]}{x^{1 + s}} \\
        &= \frac{1}{N^{s - 1}} + s \left( \int_{1}^{N}\frac{1}{x^s} - \int_{1}^{N}\frac{\{x\}}{x^{1 + s}}\right) \\
        &= \frac{1}{N^{s - 1}} + s \left( \frac{1}{s - 1} - \frac{1}{(s - 1)N^{s - 1}} - \int_{1}^{N}\frac{\{x\}}{x^{1 + s}}\right) \\
        &= 1 + \frac{1}{s - 1} - \frac{1}{(s - 1)N^{s - 1}} - s \int_{1}^{N}\frac{\{x\}}{x^{1 + s}}.
    \end{align*}
    Поскольку $s = \sigma + it$, где $\sigma > 1$, и $\abs{N^{s-1}} = N^{\sigma-1}$, то при $N \to \infty$ третье слагаемое стремится к нулю, а последнее --- к несобственному интегралу из условия Теоремы.~\newline
    Итак, мы получили, что при $\Re{s} > 1$ выполняется равенство
    \[
        \zeta(s) = 1 + \frac{1}{s-1} - s\int_{1}^{\infty} \frac{\{x\}}{x^{1+s}}dx.
    \]
    Как только мы докажем, что этот интеграл задаёт аналитическую функцию в $\Re{s} > 0$, мы получим две функции, которые аналитичны в $\Re{s} > 0$ и совпадают в $\Re{s} > 1$. По теореме единственности из этого следует, что эти две функции~совпадают~везде.~\newline
    Итак, положим
    \begin{align*}
        f_n(s) &=
        \int_{n}^{n+1} \frac{\{x\}}{x^{s+1}}dx = \int_{n}^{n+1} \frac{x-n}{x^{s+1}}dx \\
        &= \int_{n}^{n+1} \frac{1}{x^{s}}dx - n\int_{n}^{n+1} \frac{1}{x^{s+1}}dx.
    \end{align*}
    Первый интеграл аналитичен в $\CC$, второй отличается от первого просто сдвигом на единицу.~\newline
    Таким образом, $f_n(s)$ аналитична в $\CC$. Далее, при $\Re{s} = \sigma > \delta > 0$ получаем, что
    \[
        \abs{f_n(s)} 
        \le \int_{n}^{n + 1} \frac{dx}{x^{1+\sigma}} 
        \le \frac{1}{n^{1+\sigma}} 
        < \frac{1}{n^{1+\delta}}.
    \]
    Поскольку ряд $\sum_{n = 1}^{\infty} \frac{1}{n^{1 + \delta}}$ сходится, то по Признаку Вейерштрасса ряд $\sum_{n = 1}^{\infty} f_n(s)$ сходится равномерно, и потому задаёт аналитическую функцию.
\end{proof}

\begin{nproblem}
\label{prb:I-1}
    При доказательстве Теоремы~\ref{thm:I-5} мы воспользовались аналитичностью интеграла
    \[
        \int_{n}^{n+1} \frac{1}{x^{s}}dx.
    \]
    При $s \ne 1$ это просто разность степеней, и аналитичность очевидна. Докажите, что в точке $s = 1$ также есть аналитичность.
\end{nproblem}

\begin{ncorollary}
\label{crl:I-6}
    У функции $\zeta(s)$ имеется полюс первого порядка с вычетом $1$, поскольку $\Res_{1}\left(\frac{1}{s-1}\right) = 1$.
\end{ncorollary}

\begin{nlemma}
\label{lm:I-7}
    При $\Re{s} > 1$ выполнено
    \[
        \frac{\zeta'(s)}{\zeta(s)} = -s \int_{1}^{\infty} \frac{\psi(x)}{x^{1 + s}}dx.
    \]
\end{nlemma}
\begin{proof}
    Согласно пункту~\ref{thm:I-3-3} Теоремы~\ref{thm:I-3}, при $\Re{s} > 1$
    \[
        \frac{\zeta'(s)}{\zeta(s)} = -\sum_{n=1}^{\infty} \frac{\Lambda(n)}{n^s}.
    \]
    Снова используя преобразование Абеля (Лемму~\ref{lm:I-6}) с параметрами $a_n = \Lambda(n)$, $g(x) = \frac{1}{x^s}$ и тот факт, что $\sum_{n \le x} \Lambda(n) = \psi(x)$ (Лемма~\ref{lm:I-5}), получаем
    \[
        \sum_{n=1}^{N} \frac{\Lambda(n)}{n^s} 
        = \frac{\psi(N)}{N^s} + s\int_{1}^{N}\frac{\psi(x)}{x^{1+s}}dx.
    \]
    Поскольку мы знаем, что у отношения $\frac{\psi(x)}{x}$ верхний и нижний пределы ограничены, то при $\Re{s} > 1$ 
    \[
        \frac{\psi(N)}{N^{1 + s}} \to 0, \qquad N \to \infty.
    \]
    Таким образом, при $N \to \infty$ пределы выражений $\sum_{n=1}^{N} \frac{\Lambda(n)}{n^s}$ и $s\int_{1}^{N} \frac{\psi(x)}{x^{1+s}}dx$ существуют и равны, откуда и следует утверждение Леммы.
\end{proof}

\begin{nlemma}
\label{lm:I-8}
    Пусть $0 < r < 1$, $\phi \in \RR$. Тогда
    \[
        \abs{(1 - r)^3 (1 - re^{i\phi})^4 (1 - re^{2i\phi})} \le 1.
    \]
\end{nlemma}
\begin{proof}
    Положим $M = \abs{(1 - r)^3 (1 - re^{i\phi})^4 (1 - re^{2i\phi})}$. Тогда
    \begin{align*}
        \ln{M} &= 3\ln{\abs{1 - r}} + 4\ln{\abs{1 - re^{i\phi}}} + \ln{\abs{1 - re^{2i\phi}}} \\
        &= \Re{3\ln{1 - r} + 4\ln{1 - re^{i\phi}} + \ln{1 - re^{2i\phi}}} \\
        &= -\sum_{n=1}^{\infty} \frac{r^n}{n}\Re{3 + 4e^{in\phi} + e^{2in\phi}} \\
        &= -\sum_{n=1}^{\infty} \frac{r^n}{n} \left(\cos{2n\phi} + 4\cos{n\phi} + 3\right) \\
        &= -\sum_{n=1}^{\infty} \frac{r^n}{n} 2\left(\cos{n\phi} + 1\right)^2 \le 0.
    \end{align*}
    Следовательно, так как $\ln{M} \le 0$, то $M = \abs{(1 - r)^3 (1 - re^{i\phi})^4 (1 - re^{2i\phi})} \le 1$.
\end{proof}

\begin{nlemma}
\label{lm:I-9}
    При $\Re{s} > 1$, $s = \sigma + it$ справедливо неравенство
    \[
        \abs{\zeta^3(\sigma)\zeta^4(\sigma+it)\zeta(\sigma+2it)} \ge 1.
    \]
\end{nlemma}
\begin{proof}
    Положим $r = \frac{1}{p^{\sigma}}$, $e^{i\phi} = p^{-it}$. Тогда достаточно применить Лемму~\ref{lm:I-8} и формулу Эйлера (Теорему~\ref{thm:I_Euler-formula}).
\end{proof}

\begin{ntheorem}
\label{thm:I-6}
    Для всех $t \in \RR \setminus \{0\}$ выполнено
    \[
        \zeta(1 + it) \ne 0.
    \]
    В точке $t = 0$ у дзета--функции нет значения.
\end{ntheorem}
