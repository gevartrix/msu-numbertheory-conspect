Итак, на предыдущей лекции мы показали, что если мы докажем Теорему~\ref{thm:IV-12} об экспоненциальной линейной форме, то мы автоматом получим и Теорему~\ref{thm:IV-11} Линдемана--Вейерштрасса. А уже из Теоремы Линдемана--Вейерштрасса мы получим ряд следствий --- в частности, утверждение о трансцендентности числа $\pi$.

Прежде чем приступить к доказательству Теоремы~\ref{thm:IV-12}, сформулируем и докажем сперва следующую вспомогательную Лемму: 

\begin{nlemma}
\label{lm:IV-7}
    Пусть $b_0, b_1, \dots, b_m, \beta_0, \beta_1, \dots, \beta_m \in \CC$, и пусть
    \[
        \sum_{k=0}^m b_k e^{\beta_k} = 0.
    \] 
    Рассмотрим многочлены
    \begin{align*}
        f(x) = f_n(x) &= \left( x - \beta_0 \right)^n \left( x - \beta_1 \right)^{n+1} \dots \left( x - \beta_m \right)^{n+1}, \\
        g(x) = g_n(x) &= \frac{1}{n!}\sum\limits_{l \ge n} f^{(l)}(x).
    \end{align*}
    Заметим, что с некоторого $l$ слагаемые $g(x)$ станут равны нулю. Тогда
    \[
        \abs{\sum_{k=0}^m b_k g\left(\beta_k\right)} \le \frac{c^{n+1}}{n!},
    \]
    где $c = c\left(b_0, \dots, b_m, \beta_0, \dots, \beta_m\right)$ --- не зависит от $n$.
\end{nlemma}
\begin{proof}
    Положим
    \[
        F(x) = \sum_{l \ge 0} f^{(l)}(x).
    \]
    Нам требуется доказать, что
    \[
        \abs{\sum_{k=0}^m b_k F\left(\beta_k\right)} \le c^{n+1}.
    \] 
    Заметим, что
    \[
        F(0)e^{\beta_k} - F\left(\beta_k\right) = e^{\beta_k} \int_{0}^{\beta_k} e^{-z}f(z)dz,
    \]
    Осталось домножить вышеуказанное выражение на $b_k$ и просуммировать по $k$ от $0$ до $m$:
    \begin{align*}
        F(0) \sum_{k=0}^m b_ke^{\beta_k} - \sum_{k=0}^m b_k F\left(\beta_k\right) &= -\sum_{k=0}^m b_k F\left(\beta_k\right) \\
        &= \sum_{k=0}^m \left[ b_k\int_0^{\beta_k} e^{\beta_k-z}f(z)dz \right] \text{ --- хотим оценить модуль} \\
        &\le \sum_{k=0}^m \abs{b_k} e^r \cdot (2r)^{(m+1)(n+1)} \\
        &\le c^{n+1}.
    \end{align*}
    В указанных переходах подразумеваем, что
    \[
        r = \max_{0 \le k \le m}\left( \abs{\beta_k} \right), 
        \quad 
        c = (2r)^{m+1}e^r \cdot\max\left( 1,\, \sum_{k=0}^m \abs{b_k} \right).
    \]
\end{proof}

\begin{proof}[Доказательство Теоремы~\ref{thm:IV-12} об экспоненциальной линейной форме]~\newline
    Пускай \EE~--- нормальное расширение поля~\QQ, содержащее $a_0, \dots, a_m$, $\alpha_0, \dots, \alpha_m$, при этом $[E : \QQ] = \nu$, $\sigma_1, \sigma_2, \dots, \sigma_\nu$ --- все автоморфизмы \EE~над \QQ~(по аналогии с доказательству Теоремы~\ref{thm:IV-13}).~\newline
    Можно считать, что $a_0, a_1, \dots, a_m \in \ZZ_\AA$ $(\in \ZZ_\EE)$, так как существует такое $\tilde{d} \in \ZZ \setminus \{ 0 \}$, что все $\tilde{d}a_0, \tilde{d}a_1, \dots, \tilde{d}a_m \in \ZZ_\AA$. Далее, введём $d \in \NN$, что $d\alpha_0, d\alpha_1, \dots, d\alpha_m \in \ZZ_\EE$.~\newline
    Предположим противное: пусть $A(1) = 0$. Тогда продлеваем наши автоморфизмы $\sigma_1, \sigma_2, \dots, \sigma_\nu$ на $\EE[[x]]$ (снова как в доказательстве Теоремы~\ref{thm:IV-13}). То есть можно рассматривать $\left( \sigma_iA \right)(x)$.~\newline
    Так как $A(x) \in \QQ[[x]]$, то $\left( \sigma_iA \right)(x) = A(x) \ \forall i = 1, 2, \dots, \nu$. Следовательно, 
    \[
        \sum_{k=0}^m \sigma_i \left(a_k\right) e^{\sigma_i(\alpha_k)x} = (\sigma_iA)(x) = A(x).
    \]
    Таким образом, имеем:
    \[
        \sum_{k=0}^m \sigma_i\left(a_k\right)e^{\sigma_i\left(\alpha_k\right)} = A(1) = 0, \quad i = 1, 2, \dots, \nu.
    \]
    Делая теперь отсылку к Лемме~\ref{lm:IV-7}, пусть
    \begin{align*}
        f(x) = f_n(x) &= \left(x-\alpha_0\right)^n \left(x-\alpha_1\right)^{n+1} \dots \left(x-\alpha_m\right)^{n+1}, \\
        g(x) = g_n(x) &= \frac{1}{n!} \sum_{l \ge n} f^{(l)}(x).
    \end{align*} 
    Далее, положим
    \[
        I = I_n = d^{m(n+1)} \sum_{k=0}^m a_k g\left(\alpha_k\right).
    \]
    Продемонстрируем, что $I \in \ZZ_\EE$ (очевидно, что $I \in \EE$):
    \begin{align*}
        I &= \sum_{k=0}^m a_k \sum_{l \ge n} d^{m(n+1)}\frac{1}{n!}f^{(l)}\left(\alpha_k\right) \\
        &= \sum_{k=0}^m a_k \cdot (\text{целое алгебраическое число}) \\
        &= (\text{целое алгебраическое число}) \\
        &= d^{m(n+1)}a_0\frac{1}{n!}f^{(n)}\left(\alpha_0\right) + (n+1)\sum_{k=0}^m \sum_{l \ge n+1} a_k \frac{l!}{(n+1)!}d^{m(n+1)} \frac{1}{l!} f^{(l)}\left(\alpha_k\right) \\
        &= a_0 \prod_{k=1}^m \left(d\alpha_0 - d\alpha_k\right)^{n+1} + (n+1)J,\ J \in \ZZ_\EE.
    \end{align*}
    Отметим, что при втором переходе мы использовали, что
    \[
        d^{m(n+1)}f(x) = d^{-n}h(dx), \text{ где } h(t) = \left(t-d\alpha_0\right)^n\left(t-d\alpha_1\right)^{n+1} \dots \left(t-d\alpha_m\right)^{n+1}.
    \]
    Иными словами, что $h(t) \in \ZZ_\EE$. Более того, при четвёртом переходе мы заметили, что лишь одно слагаемое в $f^{(n)}\left(\alpha_i\right)$ (и только для $\alpha_0$) не равно нулю.~\newline
    Следовательно, $I \in \ZZ_\EE$, причём $I \ne 0$, если
    \[
        \gcd{n + 1,\, N\left( a_0\prod_{k=1}^m \left(d\alpha_0 - d\alpha_k\right) \right)} = 1, \text{ где } N \text{ --- норма из Определения~\ref{def:IV_extension-norm}}.
    \]
    Таких $n$ бесконечно много: например, $n + 1$ --- простое, и далее к бесконечности. Но тогда и $\sigma_i(I) \in \ZZ_\EE$, и $\sigma_i(I) \ne 0$ при ``хороших'' $n$. Но
    \begin{align*}
        \sigma_i(I) &= d^{m(n+1)} \sum_{k=0}^m \sigma_i\left(a_k\right) g_i\left(\sigma_i\left(\alpha_k\right)\right), \\
        \text{где }& f_i(x) = \left(\sigma_if\right)(x) = \left(x-\sigma_i(\alpha_0)\right)^n\left(x-\sigma_i(\alpha_1)\right)^{n+1} \dots \left(x-\sigma_i(\alpha_m)\right)^{n+1}, \\
        & g_i(x) = \left(\sigma_ig\right)(x) = \frac{1}{n!} \sum\limits_{l \ge n} f_i^{(l)}(x).
    \end{align*}
    Применяя Лемму~\ref{lm:IV-7} для $b_k = \sigma_i\left(a_k\right)$, $\beta_i = \sigma_i\left(\alpha_k\right)$, $i = 1, 2, \dots, \nu$, получаем
    \[
        \abs{\sigma_i(I)} \le d^{m(n+1)}\frac{c_i^{n+1}}{n!} \le \frac{c^{n+1}}{n!},
    \]
    где $c = d^m \max_i\left(c_i\right)$.~\newline
    Итак, все $\sigma_i(I) \in \ZZ_\EE$, $\sigma_i(I) \ne 0$ при ``хороших'' $n$ и $\sigma_i(I) \to 0,\, n \to \infty$. Следовательно,
    \begin{align*}
        N(I) &= \prod_{i=1}^{\nu} \sigma_i(I) \to 0 \text{ при } n \to \infty, \\
        N(I) &\ne 0 \text{ при ``хороших'' } n.
    \end{align*}
    Но $N(I) \in \ZZ$ --- получили противоречие с предположением, что $A(1) = 0$.
\end{proof}

Таким образом мы доказали Теорему~\ref{thm:IV-12} об экспоненциальной линейной форме и по Теореме~\ref{thm:IV-13} готовы вывести несколько следствий из Теоремы~\ref{thm:IV-11} Линдемана--Вейерштрасса.

\begin{ncorollary}
\label{crl:IV-2}
    Если $\alpha \in \AA \setminus \{ 0 \}$, то $e^\alpha \not\in \AA$.
\end{ncorollary}
\begin{proof}
    Пусть $\alpha_0 = 0$ и $\alpha_1 = \alpha$. Тогда по Теореме~\ref{thm:IV-11} $e^{\alpha_0} = 1$ и $e^{\alpha_1} = e^\alpha$ линейно независимы над \AA.
\end{proof}

\begin{ncorollary}
\label{crl:IV-3}
    Число $\pi$ является трансцендентным.
\end{ncorollary}
\begin{proof}
    Предположим противное: пусть $\pi \in \AA$. Тогда $\alpha_0 = 0$, $\alpha_1 = i\pi$. По Теореме~\ref{thm:IV-11} получаем, что $e^{\alpha_0} = 1$ и $e^{\alpha_1} = -1$ линейно независимы над \AA, но они, очевидно, линейно зависимы. Противоречие.
\end{proof}

\begin{ncorollary}
\label{crl:IV-4}
    Если $\alpha \in \AA \setminus \{ 1 \}$, то $\ln{\alpha} \not\in \AA$.
\end{ncorollary}

\begin{ncorollary}
\label{crl:IV-5}
    Если $\alpha \in \AA \setminus \{ 0 \}$, то $\sin{\alpha},\, \cos{\alpha},\, \tan{\alpha} \not\in \AA$.
\end{ncorollary}
\begin{proof}
    Докажем это Следствие только для синуса --- для остальных функций рассуждение аналогично.
    \[
        \sin{\alpha} = \frac{1}{2i}e^{i\alpha} - \frac{1}{2i}e^{-i\alpha}.
    \]
    Пусть $\sin{\alpha} \in \AA$. Тогда $i\alpha \ne -i\alpha$ и для $0, i\alpha, -i\alpha$ по Теореме~\ref{thm:IV-11} $1, e^{i\alpha}, e^{-i\alpha}$ линейно зависимы. Но они, очевидно, линейно независимы. Противоречие.
\end{proof}

\begin{ncorollary}
\label{crl:IV-6}
    Если $\beta_1, \beta_2, \dots, \beta_k \in \AA$ линейно независимы над полем \QQ, то $e^{\beta_1}, e^{\beta_2}, \dots, e^{\beta_k}$ --- алгебраически независимы над полем \AA.
\end{ncorollary}
\begin{proof}
    Пусть $f\left(x_1, x_2, \dots, x_k\right) \in \AA\left[x_1, x_2, \dots, x_k\right]$. Тогда
    \[
        f\left(e^{\beta_1}, e^{\beta_2}, \dots, e^{\beta_k}\right)
        = \sum_{\left(n_1, n_2, \dots, n_k\right)} a_{n_1 \dots n_k}\exp{n_1\beta_1 + \ldots + n_k\beta_k}.
    \]
    Во-первых, заметим, что $a_{n_1 \dots n_k} \in \AA$, а во-вторых, $n_1\beta_1 + \ldots + n_k\beta_k =: \alpha_{n_1 \dots n_k}$ --- все попарно различны, так как $\beta_1, \beta_2, \dots, \beta_k$ линейно независимы над полем \QQ.~\newline
    По Теореме~\ref{thm:IV-11} $\exp{n_1\beta_1 + \ldots +n_k\beta_k}$ линейно независимо над \AA, и, следовательно,
    \[
        f\left(e^{\beta_1}, e^{\beta_2}, \dots, e^{\beta_k}\right)
        = \sum_{\left(n_1, n_2, \dots, n_k\right)} a_{n_1 \dots n_k}\exp{n_1\beta_1 + \ldots + n_k\beta_k} \ne 0.
    \]
\end{proof}

\thispagestyle{supplementary}

\begin{thebibliography}{3}
    \bibitem{GNSh1984}
        Галочкин А.И., Нестеренко Ю.В., Шидловский А.Б. 
        \textit{Введение в теорию чисел, 4.\S1--4.\S5}. 
        Издательство МГУ, 1984.

    \bibitem{Shidlovsky1987}
        Шидловский А.Б. 
        \textit{Трансцендентные числа}. 
        Издательство ``Наука'', 1987.

    \bibitem{Gelfond1952}
        Гельфонд А.О.  
        \href{https://ikfia.ysn.ru/wp-content/uploads/2018/01/Gelfond1952ru.pdf}{\textit{Трансцендентные и алгебраические числа}}. 
        Государственное издательство технико--теоретической литературы, 1952.
\end{thebibliography}

