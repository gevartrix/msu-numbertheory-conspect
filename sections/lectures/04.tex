\begin{proof}[Доказательство Теоремы~\ref{thm:I-6}]
    Предположим противное: пусть $\zeta\left(1 + it_0 \right) = 0$. Тогда при $\sigma \to 1^+$
    \begin{align*}
        \zeta^3(\sigma) \zeta^4\left(\sigma + it_0 \right) \zeta\left(\sigma + 2it_0 \right) 
        &= \bigO{\frac{1}{(\sigma - 1)^3} \cdot (\sigma - 1)^4 \cdot 1} \\
        &= \bigO{\sigma - 1}, \quad \sigma \to 1^+.
    \end{align*}
    Поясним:
    \begin{itemize}[label=$\circ$]
        \item
            Первый множитель взялся из того, что $\zeta(\sigma) \to +\infty$ при $\sigma \to 1+$, а точнее, $\zeta(\sigma) = \bigO{\frac{1}{\sigma-1}}$, т.к. у функции $\zeta(\sigma)$ полюс порядка $1$;
        \item
            второй множитель получается по аналитическому продолжению: 
            \[
                \zeta\left( 1 + it_0 \right) = 0 \ 
                \Rightarrow \ 
                \zeta\left( \sigma + it_0 \right) = \bigO{\sigma - 1};
            \]
        \item
            наконец, последний множитель равен $1$, т.к. $\zeta\left( 1 + 2it_0 \right)$ --- какая-то константа, и полюса там нет из аналитичности функции в полуплоскости $\Re{s} > 0$ везде, кроме $1$.
    \end{itemize}
    Итак, мы получили, что 
    \[
        \zeta^3(\sigma) \zeta^4\left(\sigma + it_0\right) \zeta\left(\sigma + 2it_0\right) = \bigO{\sigma - 1}, \quad \sigma \to 1^+,
    \]
    но согласно Лемме~\ref{lm:I-9}, модуль левой части выражения выше $\ge 1$ при любом $\sigma > 1$. Получили противоречие.
\end{proof}

\begin{remark}
    Из Леммы~\ref{lm:I-9} также можно ещё одним способом получить, что в полуплоскости $\Re{s} > 1$ у $\zeta$--функции нет корней: если бы существовал корень $s = \sigma + it$, то 
    \[
        \abs{\zeta^3(\sigma) \zeta^4(s) \zeta(\sigma + 2it)} \ge 1,
    \]
    где множитель $\zeta^4(s) = 0$. Снова получаем противоречие с Леммой~\ref{lm:I-9}.
\end{remark}

\begin{nlemma}
\label{lm:I-10}
    При $\Re{s} \ge 1$ следующая функция аналитична:
    \[
        \frac{\zeta'(s)}{\zeta(s)} + \frac{1}{s-1}.
    \]
\end{nlemma}
\begin{proof}
    Нам уже известно, что при $\Re{s} > 1$ $\frac{\zeta'(s)}{\zeta(s)}$ и $\frac{1}{s-1}$ --- аналитические функции. Мы также доказали, что $\zeta(s) = \frac{f(s)}{s-1}$, где $f(s)$ точно аналитична при $\Re{s} > 0$ и $f(s) \ne 0$ при $\Re{s} \ge 1$.~\newline
    Отсюда следует, что 
    \[
        \frac{\zeta'(s)}{\zeta(s)} = \frac{f'(s)}{f(s)} - \frac{1}{s-1},
    \]
    где $f$ аналитична при $\Re{s} > 0$. А значит, что и $f'$ тоже является аналитичной. Осталось заметить, что в полуплоскости $\Re{s} \ge 1$ у знаменателя нет нулей, поэтому Лемма доказана.
\end{proof}


\subsection{Доказательство АЗРПЧ}
\label{subsec:5_Prime-number-theorem-proof}

\begin{ndefinition}
\label{def:I_F-function}
    Определим следующую функцию:
    \[
        F(s) := -\frac{1}{s}\frac{\zeta'(s)}{\zeta(s)} - \frac{1}{s-1}.
    \]
\end{ndefinition}

\begin{nlemma}
\label{lm:I-11}
    Для выше определённой функции справедливы следующие утверждения
    \begin{enumerate}
        \item $F(s)$ аналитична в $\Re{s} \ge 1$.
        \item $F(s) = \int_{1}^{+\infty} \frac{\psi(x) - x}{x^{1+s}}dx$ при $\Re{s} > 1$.
    \end{enumerate}
\end{nlemma}
\begin{proof}
    \hfill
    \begin{statesp}
        \item
            \begin{align*}
                F(s) &= -\frac{1}{s}\left( \frac{\zeta'(s)}{\zeta(s)} + \frac{s}{s-1} \right) \\
                &= -\frac{1}{s}\left( \frac{\zeta'(s)}{\zeta(s)} + \frac{1}{s-1} + 1 \right),
            \end{align*}
            где $-\frac{1}{s}$ аналитична, а второй множитель аналитичен по Лемме~\ref{lm:I-10}.
        \item
            При $\Re{s} > 1$ имеем
            \[
                \frac{\zeta'(s)}{\zeta(s)} = 
                -s\int_{1}^{+\infty} \frac{\psi(x)dx}{x^{1+s}},
                \qquad
                \frac{1}{s-1} = \int_{1}^{+\infty}\frac{dx}{x^s}.
            \]
            Оба интеграла сходятся абсолютно, поэтому мы можем их складывать:
            \[
                F(s) = \int_{1}^{+\infty}\frac{\psi(x)dx}{x^{1+s}} - \int_{1}^{+\infty} \frac{dx}{x^s} = \int_{1}^{+\infty} \frac{\psi(x)-x}{x^{1+s}}dx.
            \]
    \end{statesp}
\end{proof}

\begin{ntheorem}
\label{thm:I-7}
    В интегральном представлении $F(s)$ в полуплоскости $\Re{s} > 1$ можно перейти к пределу:
    \[
        F(1) = \int_{1}^{+\infty} \frac{\psi(x) - x}{x^2}dx.
    \]
\end{ntheorem}

Допустим, что Теорема~\ref{thm:I-7} верна. Тогда можно доказать следующую примечательную Лемму, которая в совокупности с Леммой~\ref{lm:I-1} мгновенно докажет Асимптотический Закон Распределения Простых Чисел:
\begin{nlemma}
\label{lm:I-12}
    Если интеграл $\int_{1}^{+\infty} \frac{\psi(x) - x}{x^2}dx$ сходится (а это будет следовать из Теоремы~\ref{thm:I-7}), то
    \[
        \psi(x) \sim x.
    \]
\end{nlemma}
\begin{proof}
    Возьмём $\epsilon >0$:
    \[
        \int_x^{(1+\epsilon)x} \frac{\psi(u) - u}{u^2}du 
        \ge \epsilon x \frac{\psi(x) - (1 + \epsilon)x}{(1 + \epsilon)^2x^2} 
        = \frac{\epsilon}{\left(1 + \epsilon\right)^2x^2}\left(\frac{\psi(x)}{x} - (1 + \epsilon) \right).
    \]
    Из сходимости интеграла слева при фиксированном $\epsilon$ следует, что
    \[
        \int_x^{(1+\epsilon)x} \frac{\psi(u) - u}{u^2}du \xrightarrow{x \to \infty} 0.
    \]
    Следовательно, $\limsup_{x\to\infty}\left(\frac{\epsilon}{(1 + \epsilon)^2}\left( \frac{\psi(x)}{x} - (1 + \epsilon) \right)\right) \le 0$ при фиксированном значении $\epsilon$. Отсюда $\limsup_{x\to\infty} \left(\frac{\psi(x)}{x}\right) \le 1 + \epsilon$, а т.к. это верно для любого $x$, то 
    \[
        \limsup_{x\to\infty} \left(\frac{\psi(x)}{x}\right) \le 1.
    \]
    И наоборот --- меняя знак неравенства, получаем $\limsup_{x\to\infty} \left(\frac{\psi(x)}{x}\right) \ge 1 - \epsilon$, а т.к. это выполняется для любого $\epsilon$, то 
    \[
        \limsup_{x\to\infty} \left(\frac{\psi(x)}{x}\right) \ge 1.
    \]
\end{proof}

\begin{proof}[Доказательство Теоремы~\ref{thm:I-7}]
    Положим функцию
    \[
        F_T(s) := \int_{1}^{T} \frac{\psi(x) - x}{x^{1+s}}dx, \, T > 1.
    \]
    Поскольку $\int_{n}^{n+1} \frac{dx}{x^s}$ --- целая функция, так как на отрезке $[n, n+1]$ функция $\psi(x)$ является постоянной, то $\int_{1}^{T} \frac{\psi(x) - x}{x^{1+s}}dx$ является суммой целых функций вида $\int_{n}^{n+1}\frac{dx}{x^s}$. Следовательно, функция $F_T(s)$ --- также целая.~\newline
    Нужно показать, что $F_T(1) \to F(1)$ при $T \to \infty$. По определению предела, возьмём $\epsilon > 0$ и рассмотрим следующий интеграл
    \[
        I(T) = \frac{1}{2\pi i}\int_{\Gamma} \left( F(1+s) - F_T(1+s) \right) T^s \left(\frac{s}{R^2} + \frac{1}{s}\right)ds, \ \text{ где } R = \frac{1}{\epsilon}.
    \]
    Нам известно, что $F(s)$ аналитична в $\Re{s} \ge 1$, а значит $F(1+s)$ аналитична в $\Re{s} \ge 0$. То есть, она аналитична на отрезке $[-iR, iR]$ (см. Рис. \ref{fg:I-1}).
    \begin{figure}[h]
        \centering
        % Plot for the Gamma-contour used in the proof of Theorem I.7
% See Lecture 4
\tikz[scale=0.7]{
    % Sets the background grid
    \draw[gray,ultra thin,step=1cm] 
        (-3.9, -3.9) grid (3.9, 3.9);
    % Sets the axes
    \draw[thick,->] 
        (-3.9, 0) -- (3.9, 0) 
        node[anchor=south east]{$\mathrm{Re}$};
    \draw[thick,->] 
        (0, -3.9) -- (0, 3.9) 
        node[anchor=north west]{$\mathrm{Im}$};
    % Draws the origin point
    \filldraw 
        (0, 0) circle (2.5pt) 
        node[anchor=south east]{$O$};

    % Constructs the contour itself
    \draw[ultra thick,domain=-105:105] 
        plot ({3*cos(\x)}, {3*sin(\x)}) 
        node[anchor=north east]{$\Gamma$};
    \draw 
        (0, 0) -- (60:3) 
        node[midway, above]{$R$};
    \draw[ultra thick,red] 
        ({3*cos(105)}, {-3*sin(105)}) -- ({3*cos(105)}, {3*sin(105)});
    % Marks all the points on the contour
    \filldraw 
        (0, -3) circle (2.5pt) 
        node[anchor=north east]{$-iR$};
    \filldraw 
        (0, 3) circle (2.5pt) 
        node[anchor=south east]{$iR$};
    \filldraw 
        ({3*cos(60)}, {3*sin(60)}) 
        circle (2.5pt);
    \filldraw[red] 
        ({3*cos(105)}, 0) circle (2.5pt) 
        node[anchor=south east]{$-h$};
}

        \caption{Контур $\Gamma$}
        \label{fg:I-1}
    \end{figure}

    \noindent Далее, если $F(s)$ аналитична в точке, то она аналитична в некоторой окрестности этой точки. Применяя это к каждой точке нашего отрезка, получаем его покрытие открытыми кругами и выделяем конечное подпокрытие по компактности $[-iR, iR]$.~\newline
    Теперь выбираем такое $h$ так, чтобы прямоугольник, длина которого есть $[-iR, iR]$, а ширина есть $h$, лежал внутри объединения полученных кругов. Очевидно, что $h = h(\epsilon)$ (см. Рис.~\ref{fg:I-2}). То есть мы подобрали такое $h$, чтобы $F(1+s)$ стала аналитична на нарисованном контуре.
    \begin{figure}[ht]
        \centering
        % Illustration of the choice of `h` in the proof of Theorem I.7
% See Lecture 4
\tikz[scale=0.7]{
    % Sets the background grid
    \draw[gray,ultra thin,step=1cm] 
        (-1.5, -3.9) grid (1.5, 3.9);
    % Sets the imaginary axis
    \draw[thick,->] 
        (0, -3.9) -- (0, 3.9) 
        node[anchor=north west]{$\mathrm{Im}$};
    % Draws the origin point
    \filldraw 
        (0, 0) circle (2.5pt) 
        node[anchor=south east]{$O$};

    % Constructs the line segment and marks its points
    \draw[ultra thick] 
        (0, -3) -- (0, 3);
    \filldraw 
        (0, -3) circle (2.5pt) 
        node[anchor=north east]{$-iR$};
    \filldraw 
        (0, 3) circle (2.5pt) 
        node[anchor=south east]{$iR$};

    % Constructs a bunch of circles
    \fill[fill=purple, opacity=0.12] (0, 2.7) circle (0.5);
    \fill[fill=purple, opacity=0.12] (0, 2.2) circle (0.55);
    \fill[fill=purple, opacity=0.12] (0, 1.3) circle (0.8);
    \fill[fill=purple, opacity=0.12] (0, 0.4) circle (1);
    \fill[fill=purple, opacity=0.12] (0, -0.4) circle (0.9);
    \fill[fill=purple, opacity=0.12] (0, -1.1) circle (0.6);
    \fill[fill=purple, opacity=0.12] (0, -1.9) circle (0.88);
    \fill[fill=purple, opacity=0.12] (0, -2.6) circle (0.64);

    % Draws the required `h`
    \draw[red,thick] 
        (0, 2.49) -- (0.46, 2.49) 
        node[anchor=south west]{$h$};
    % Constructs the required rectangle
    \fill[fill=blue, opacity=0.24] (0, -3) rectangle (0.46, 3);
}

        \caption{Выбор значения $h$}
        \label{fg:I-2}
    \end{figure}

    Таким образом, возвращаясь к интегралу $I(T)$:
    \[
        I(T) = \frac{1}{2\pi i}\int_{\Gamma} \left( F(1+s) - F_T(1+s) \right) T^s \left(\frac{s}{R^2} + \frac{1}{s}\right)ds,
    \]
    функция $F(1+s)$ является аналитичной в области (по построению), следовательно $F_T(1+s)$ --- везде целая, $T^s$ --- целая (экспонента), $\frac{s}{R^2}$ --- целая, а $\frac{1}{s}$ --- полюс порядка $1$ в нуле. Следовательно, по теореме Коши о вычетах\footnote{Эту теорему из курса ТФКП нужно знать!}
    получаем искомый результат:
    \[
        I(T) = \left(F(1) - F_T(1)\right)T^0 = F(1) - F_T(1).
    \]
\end{proof}

\begin{nlemma}
\label{lm:I-13}
    В зависимости от $\sigma = \Re{s}$ справедливы следующие утверждения:
    \begin{align*}
        \sigma > 0\colon & \abs{F(1+s) - F_T(1+s)} \le A\frac{T^{-\sigma}}{\sigma}, \\
        \sigma < 0\colon & \abs{F_T(1+s)} \le A\frac{T^{-\sigma}}{-\sigma},
    \end{align*}
    где $A$ такое, что $\abs{\frac{\psi(x)}{x} - 1} \le A$ при $x \ge 1$.
\end{nlemma}
\begin{proof}
    \begin{align*}
        \sigma > 0\colon& \abs{F(1+s) - F_T(1+s)} 
          = \abs{\int_T^{+\infty}\frac{\psi(x)-x}{x^{2+s}}dx} 
          \le A\int_{T}^{+\infty}\frac{dx}{x^{1+\sigma}} 
          = A\frac{T^{-\sigma}}{\sigma};\\
        \sigma < 0\colon& \abs{F_T(1+s)} 
          = \abs{\int_1^T\frac{\psi(x)-x}{x^{2+s}}dx} 
          \le A\int_1^T\frac{dx}{x^{1+\sigma}} 
          = A\frac{T^{-\sigma}}{-\sigma}.
    \end{align*}
\end{proof}
