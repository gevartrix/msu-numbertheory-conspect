\begin{center}
	\Large \bf{Предисловие}~\\
\end{center}

\tab \textbf{Внимание!} Это не курс лекций и не методичка, а всего лишь конспект, набранный в вёрстке \LaTeX~{}и не претендующий на окончательную истину. В данном документе не исключены опечатки. Использовать на свой страх и риск. Авторы не несут ответственности за успешность подготовки по данному материалу, а также за его использование в качестве "шпоры".

\tab Данный конспект по теории чисел состоит из $14$-ти лекций, прочитанных Олегом Николаевичем Германом — доцентом кафедры теории чисел. Курс был прочитан на $7$-ом семестре четвёртого курса мехмата МГУ осенью $2017$ года. Он состоит из трёх больших раздедов:
\begin{enumerate}[nolistsep]
	\item Асимптотический закон распределения простых чисел:
			$$\pi(x) = \sum_{p \leq x} 1 \sim \frac{x}{\ln x}.$$
	\item Теорема Дирихле о простых числах в арифметических прогрессиях:
			Если $(l, m) = 1$, то существует бесконечное количество таких простых $p$, что $p \equiv l \mod m$.
	\item Теоремы о том, что  $e$ и $\pi$ — иррациональные и трансцендентные числа.
\end{enumerate}

\tab Конспект был подготовлен и за\TeX ан студентами Артемием Соколовым, группа $405$ (нечётные лекции) и Артемием Геворковым, группа $402$ (чётные лекции). За основу был взят конспект Юлии Зайцевой. Также в перспективе планируется добавить решения всех упражнений из курса.

\tab Данная версия документа была скомпилирована \today~{}Последняя версия .PDF, а также все исходные файлы всегда будут доступны в репозитории по (кликабельной) ссылке:
\begin{center}
\href{https://github.com/arvego/numbertheory-sem7}{https://github.com/arvego/numbertheory-sem7}
\end{center}
Если найдена ошибка или опечатка — пожалуйста, сообщите нам.\\



\tab Спасибо Юлии Зайцевой, Виталию Лобачевскому, Всеволоду Гусеву, Кириллу Сосову, Сергею Джунусову, Айку Эминяну, Александру Думаревскову и команде Алгебрача за поиск ошибок и помощь в оформлении данного материала.