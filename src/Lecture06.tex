\begin{lemma} \label{l6_lm1}
	Пусть $\eta_1, \dots, \eta_r$ — произвольный набор корней из $1$ степеней $d_1, \dots, d_r$ соответственно (т.е. $\eta_i^{d_i} = 1$).
	Тогда $\exists! \chi: \ \chi(g_i) = \eta_i$.
\end{lemma}
\begin{pf}
	Для $(a,m)=1$ полагаем $\chi(a) = \eta_1^{\alpha_1} \dots \eta_r^{\alpha_r}$, где $\overline{a} = \overline{g}_1^{\alpha_1} \dots \overline{g}_r^{\alpha_r}$. Для $(a,m) \ne 1$ полагаем $\chi(a) = 0$. Достаточно проверить, что если
	$(a,m)=1, \, (b,m)=1$, то $\chi(ab) = \chi(a)\chi(b)$.\\
	Пусть $\overline{a} = \overline{g}_1^{\alpha_1} \dots \overline{g}_r^{\alpha_r}, \, \overline{b} = \overline{g}_1^{\beta_1} \dots \overline{g}_r^{\beta_r}, \, \overline{c} = \overline{g}_1^{\gamma_1} \dots \overline{g}_r^{\gamma_r}$, где $0 \leq \alpha_i, \beta_i, \gamma_i \leq d_1 - 1, \, i=1 \dots r$.
	Тогда $\gamma_i \equiv \alpha_i + \beta_i \ (\mathrm{mod} \; d_i)$. Следовательно, т.к. $\eta_i$ — корень из $1$ степени $d_i$, получаем
	$$\chi(a)\chi(b) = \eta_1^{\alpha_1} \dots \eta_r^{\alpha_r} \cdot \eta_1^{\beta_1} \dots \eta_r^{\beta_r} = \eta_1^{\gamma_1} \dots \eta_r^{\gamma_r} = \chi(ab).$$
\end{pf}

\begin{lemma} \label{l6_lm2}
	Если $a \not \equiv 1 \ (\mathrm{mod} \; m)$, то $\exists \chi: \ \chi(a) \ne 1$.
\end{lemma}
\begin{pf}
	Очевидно из Леммы \ref{l6_lm1}: если $(a,m) \ne 1$, то все характеры подходят; если $(a,m) = 1$, то $\overline{a}=\overline{g}_1^{\alpha_1} \dots \overline{g}_r^{\alpha_r} \ (\mathrm{mod} \; m), \, 0 \leq \alpha_i \leq d_i-1$. Т.к. $a \not\equiv 1 \ (\mathrm{mod} \; m)$, то $\exists \alpha_i \ne 0$, можно положить, что это $\alpha_1 > 0$.\\
	Положим $\chi\left( g_1 \right) = \eta_1 = e^{\frac{2\pi i}{d_1}}, \, \chi\left( g_j \right) = \eta_j = 1, \ \forall j = 2, \dots, r$.		Тогда, т.к. по Лемме \ref{l6_lm1} характер существует, то $\chi(a) = e^{\frac{2\pi i}{d_1}\alpha_1} \ne 1$.
\end{pf}

\begin{definition}
	Характер $\chi_0$, где $\chi_0(a) = \begin{cases}
		1, & (a,m)=1, \\
		0, & (a,m) \ne 1
	\end{cases}$ называется \textit{главным характером}.
\end{definition}

Ясно, что $\chi \cdot \chi_0 = \chi$, где операция $\cdot$ — поточечное перемножение функций. Для любого $\chi$ существует обратное $\chi^{-1}: \ \chi^{-1}(a) = \begin{cases}
	\chi(a)^{-1}, & \chi(a) \ne 0, \\
	0, & \chi(a) = 0.
\end{cases}$ В общем, ясно, что характеры образуют группу.

\begin{problem} \label{l6_task}
	Доказать, что группа характеров изоморфна $\mathbb{Z}_m^\ast$.
\end{problem}

Характеров по модулю $m$ ровно $\phi(m)$ штук (следует из Леммы \ref{l6_lm1}, $d_1 \dots d_r = \lvert \mathbb{Z}_m^\ast \rvert = \phi(m)$).

\begin{lemma} \label{l6_lm3}
	Справедливы следующие равенства
	\begin{enumerate}[nolistsep]
		\item[1)] $\displaystyle \sum\limits_{a=1}^m \chi(a) =
			\begin{cases}
				\phi(m), & \text{если } \chi=\chi_0, \\
				0, & \text{иначе.}
			\end{cases}$
		\item[2)] $\displaystyle \sum\limits_\chi \chi(a) =
		\begin{cases}
			\phi(m), & \text{если } a \equiv 1 \ (\mathrm{mod} \; m), \\
			0, & \text{иначе.}
		\end{cases}$
	\end{enumerate}
\end{lemma}
\begin{pf}
	По порядку.
	\begin{enumerate}[nolistsep]
		\item[1)] Если $\chi = \chi_0$, то всё понятно.\\
			Если $\chi \ne \chi_0,$ то $\exists b \in \mathbb{Z}: \ \chi(b) \ne 0, \, 1$. Положим $\displaystyle s = \sum\limits_{a=1}^m \chi(a)$, тогда $\displaystyle s \chi(b) = \sum\limits_{a=1}^m \chi(ab) = \sum\limits_{a=1}^m \chi(a) = s \ \Rightarrow \ s = 0$.
		\item[2)] Если $a \equiv 1 \ (\mathrm{mod} \; m)$, то всё понятно.\\
			Если $(a,m) \ne 1$, то сумма из нулей равна нулю (действительно).\\
			Если $(a,m) = 1$ и $a \not\equiv 1 \ (\mathrm{mod} \; m)$, то по Лемме \ref{l6_lm2} можно взять характер $\chi_1: \ \chi_1(a) \ne 0, \, 1.$
			Положим $\displaystyle s = \sum\limits_\chi \chi(a)$, тогда $\displaystyle s \chi_1(a) = \sum\limits_\chi \chi(a)\chi_1(a) = \sum\limits_\chi \chi(a) = s \ \Rightarrow \ s = 0$.
	\end{enumerate}
\end{pf}

\begin{corollary} \label{l6_cor}
	Если $\chi \ne \chi_0$, то $\displaystyle \left| \sum\limits_{n=1}^m \chi(n) \right| \leq \phi(m)$.
\end{corollary}~\\

\subsection{$L$-функции Дирихле}
Пусть $m \geq 2, \, \chi$ — характер по модулю $m$.
\begin{definition}
	$\displaystyle L(s, \, \chi) = \sum\limits_{n=1}^\infty \frac{\chi(n)}{n^s}$ называется \textit{$L$-функцией Дирихле}.
\end{definition}

\begin{lemma} \label{l6_lm4}
	При $\Re(s) > 1$
	\begin{enumerate}[nolistsep]
		\item[1)] Ряд $\displaystyle \sum\limits_{n=1}^\infty \frac{\chi(n)}{n^s}$ сходится абсолютно, задаёт аналитическую функцию;
		\item[2)] $\displaystyle L'(s, \, \chi) = -\sum\limits_{n=1}^\infty \frac{\chi(n)\ln(n)}{n^s}$;
		\item[3)] $\displaystyle L(s, \, \chi) \ne 0$ и $\frac{L'(s,\,\chi)}{L(s,\,\chi)} = -\sum\limits_{n=1}^\infty \frac{\chi(n)\Lambda(n)}{n^s}$.
	\end{enumerate}
\end{lemma}
\begin{pf}
	Доказательство этой теоремы очень схоже с доказательством теоремы \ref{l2_th3}. Напомним, что $s = \sigma + it$.
	\begin{enumerate}[nolistsep]
		\item[1)] $\displaystyle \left| \frac{\chi(n)}{n^s} \right| \leq \frac{1}{n^\sigma} \ \Rightarrow$ сходится абсолютно при $\sigma > 1$. Но для аналитичности предела нам необходима равномерная сходимость. В области $\displaystyle \Omega_\delta = \{ \Re(s) > 1+\delta \} \ \left| \frac{\chi(n)}{n^s} \right| \leq \frac{1}{n^\sigma} \leq \frac{1}{n^{1+\delta}}$ — общий член сходящегося ряда. значит, по признаку Вейерштрасса в $\Omega_\delta$ наша последовательность равномерна. Следовательно, по теореме \ref{l2_th_Weierstrass} (Вейерштрасса) ряд сходится к аналитической функции.
		\item[2)] В первом пункте мы воспользовались теоремой Вейерштрасса, которая, в частности, гласит, что наш ряд можно почленно дифференцировать.
		\item[3)] $\displaystyle L(s,\,\chi)\cdot\sum\limits_{n=1}^\infty \frac{\chi(n)\Lambda(n)}{n^s} = \sum\limits_{k=1}^\infty \frac{\chi(k)}{k^s} \sum\limits_{n=1}^\infty \frac{\chi(n)\Lambda(n)}{n^s} = \sum\limits_{k,n \in \mathbb{N}} \frac{\chi(k)\chi(n)\Lambda(n)}{(kn)^s} = \sum\limits_{k,n \in \mathbb{N}} \frac{\chi(kn)\Lambda(n)}{(kn)^s} = \sum\limits_{\substack{n \in \mathbb{N} \\ d \vert n}} \frac{\chi(n)\Lambda(d)}{n^s} = \sum\limits_{n \in \mathbb{N}} \frac{\chi(n) \ln(n)}{n^s} = -L'(s,\,\chi)$.\\
			Итак, получили $\displaystyle L(s,\,\chi) \cdot \sum\limits_{n=1}^\infty \frac{\chi(n)\Lambda(n)}{n^s} = -L'(s,\,\chi)$.\\
			Если $s_0$ — ноль порядка $k \in \mathbb{N}$, то порядок нуля левой части будет больше или равен $0$, т.к. мы умножаем на некую аналитическую функцию. Но порядок нуля правой части равен $k-1$. Противоречие.\\
Осталось показать, почему $L(s, \, \chi) \not\equiv 0$: $\displaystyle \left| \sum\limits_{n=2}^\infty \right| \leq \sum\limits_{n=2}^\infty \frac{1}{n^\sigma} = \frac{1}{\alpha^\sigma}\sum\limits_{n=2}^\infty \frac{1}{\left( \frac{n}{2}\right)^\sigma}$ — первый множитель стремится к $0$, второй множитель ограничен некой константой $\displaystyle C \ \Rightarrow \ \sum\limits_{n=1}^\infty \frac{\chi(n)}{n^s} = 1+\sum\limits_{n=2}^\infty \frac{\chi(n)}{n^s}$. Второе слагаемое по модулю стремится к $0$ при $\sigma \to \infty$. Следовательно, $L(s, \, \chi) \ne 0$ для некоторого $s$.
	\end{enumerate}
\end{pf}